\documentclass{notes}
\usepackage{math55,quiver,hyperref,lipsum}

\usepackage{setspace}
\onehalfspacing

\makeheader{Rushil Mallarapu}{Thinking Categorically}

\begin{document}

\section{Introduction}

\vspace{-3ex}

\subsection{Housekeeping}

\vspace{-3ex}
Hello everyone, thank you all for coming.

\vspace{-8.5ex}
Some quick housekeeping: This Math Table talk is called ``Thinking in Categories: What Abstract Nonsense Can Do For You.'' My name is Rushil Mallarapu and I’m a first-year undergraduate thinking about concentrating in mathematics. I’ll be sending out notes from this talk so that you can reference those if you find these topics interesting, and I’ll include my email\footnote{\href{mailto:rushil\_mallarapu@college.harvard.edu}{rushil\_mallarapu@college.harvard.edu}} on that, so if you have any questions, comments, or just want to talk about category stuff, feel free to send me an email.

\vspace{-8.5ex}
A huge thank you to the Harvard Math Table for organizing this wonderful series, and giving me the opportunity to present this to you.

\vspace{-8.5ex}
The talk should take about 25 minutes, and we’ll have some time for questions at the end before we break to get food — if you have any clarifications, feel free to ask whenever, but I’m going to try and not get carried away.

\subsection{Motivation}

\vspace{-3ex}
Today I will be talking about category theory. For those of you who have never heard of category theory, I hope you are excited. For those of you who have heard of category theory, I hope to show you why I care, not just about category theory, but a categorical style of thinking.

\vspace{-8.5ex}
Broadly speaking, there are three big reasons category theory is worthwhile to dive into. First, it is beautiful. Second, it is useful. And third, it is everywhere. In these 25 minutes, my goal is to convince you of these claims, and hopefully give you enough intuition to try and adopt this categorical perspective in your everyday mathematical pursuits.

\subsection{Disclaimers \& Outline}

\vspace{-3ex}
Before we get started, some short disclaimers. There are so many directions I could go with this talk, so I've had to be judicious in cutting out certain proofs, pitfalls, and more formal definitions. Learning the difference between small and locally small categories, for instance, is certainly useful, but my goal is to give you an intuitive sense for what is going on. If you have any additional questions or want to see certain things proved out, then feel free to ask me or check your favorite category theory book.

\vspace{-8.5ex}
So, the plan for today is first, to start by talking through some basic definitions, and second, to do a worked example of ``categorical thinking.'' We're going to cover some really cool things really quickly, so hold on for the ride!

\section{Basic Definitions}

\vspace{-3ex}

\subsection{Categories}

\vspace{-3ex}

\subsection{Functors}

\vspace{-3ex}

\subsection{Natural Transformations}

\vspace{-3ex}

\subsection{Representations \& Yoneda}

\vspace{-3ex}

\subsection{Limits}

\vspace{-3ex}

\section{Categorical Thinking}

\vspace{-3ex}

\subsection{Topological Spaces}

\vspace{-3ex}

\subsection{Colimits in $\text{Group}$}

\vspace{-3ex}

\subsection{Fundamental Group}

\vspace{-3ex}

\subsection{Van Kampen Theorem}

\vspace{-3ex}\lipsum[2]


\vspace{-8.5ex}
Thank you so much, and I would love to take any questions!

\end{document}
